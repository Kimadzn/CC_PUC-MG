\chapter{Introdução}
\label{intro}



O home office, que não era tão popular antes de 2020, está atualmente em evidência em todo o mundo, principalmente devido à pandemia de COVID-19. A mudança para esse formato de trabalho trouxe desafios e vantagens para as empresas e seus funcionarios. Apesar da necessidade de se ajustar a novos métodos de liderança e comunicação, os pontos positivos incluem maior flexibilidade e economia de recursos.

Por outro lado, o trabalho remoto apresenta desafios, como a ausência de contato com outras pessoas, e o foco por se tratar de um ambiente cômodo como sua casa. Além disso, torna-se cada vez mais complexo conciliar a vida pessoal e o trabalho, já que o ambiente de descanso se torna seu ambiente de trabalho. 

\textit{"No âmbito das empresas, o trabalho remoto pode representar uma redução de custos, uma vez que não é necessário manter um grande escritório para acomodar todos os funcionários. Essa modalidade permite também a contratação de profissionais talentosos de diferentes partes do mundo, sem a necessidade de deslocamento" 
}
\textbf{-Disponivel em https://www.undb.edu.br/blog/profissoes-home-office-oportunidades-e-desafios-do-trablho-remoto#:~:text=No%20âmbito%20das%20empresas%2C%20o,sem%20a%20necessidade%20de%20deslocamento.}



Este trabalho está organizado da seguinte forma. A seção~\ref{sec:Obj} asdadas  dasda. O capítulo~\ref{revisao} apresenta o referencial teórico usado neste trabalho. O capítulo~\ref{metodo} descreve os procedimentos metodológicos  ...


\section{Objetivos}
%\addcontentsline{toc}{chapter}{Objetivos}
\label{sec:Obj}

O objetivo geral deste projeto é ...

\subsection{Objetivos específicos}

Os objetivos específicos deste projeto são:
\begin{enumerate}
    \item asdasd
    \item asdasd
    \item asd
\end{enumerate}